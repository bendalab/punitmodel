\documentclass[12pt,a4paper]{article}

\title{P-unit model by Henriette Walz \& Alexander Ott}
\author{Alexandra Rudnaya, Jan Grewe, Jan Benda}
\date{June 2022}

%%%%% layout %%%%%%%%%%%%%%%%%%%%%%%%%%%%%%%%%%%%%%%%%%%%%%%%%%%%%%%%%%%%%%%%
\usepackage[left=20mm,right=20mm,top=20mm,bottom=20mm]{geometry}
%\setcounter{secnumdepth}{-1}

%%%%% units %%%%%%%%%%%%%%%%%%%%%%%%%%%%%%%%%%%%%%%%%%%%%%%%%%%%%%%%%%%%%%%%%
\usepackage[mediumspace,mediumqspace,Gray,amssymb]{SIunits}      % \ohm, \micro

%%%%% math %%%%%%%%%%%%%%%%%%%%%%%%%%%%%%%%%%%%%%%%%%%%%%%%%%%
\usepackage{textcomp}
\usepackage{array}
\usepackage{amsmath}
\usepackage{amssymb}

%%%%% equation references %%%%%%%%%%%%%%%%%%%%%%%%%%%%%%%%%%%%%%%%%%%%%%%%%%%
\renewcommand{\eqref}[1]{(\ref{#1})}
\newcommand{\eqn}{Eq.}
\newcommand{\Eqn}{Eq.}
\newcommand{\eqns}{Eqs.}
\newcommand{\Eqns}{Eqs.}
\newcommand{\eqnref}[1]{\eqn~\eqref{#1}}
\newcommand{\Eqnref}[1]{\Eqn~\eqref{#1}}
\newcommand{\eqnsref}[1]{\eqns~\eqref{#1}}
\newcommand{\Eqnsref}[1]{\Eqns~\eqref{#1}}
 
%%%%% hyperef %%%%%%%%%%%%%%%%%%%%%%%%%%%%%%%%%%%%%%%%%%%%%%%%%%%%%%%%%%%%%%%
\usepackage{xcolor}
\usepackage[breaklinks=true,colorlinks=true,citecolor=blue!30!black,urlcolor=blue!30!black,linkcolor=blue!30!black]{hyperref}

%%%%% notes %%%%%%%%%%%%%%%%%%%%%%%%%%%%%%%%%%%%%%%%%%%%%%%%%%%%%%%%%%%%%%%%
\newcommand{\note}[2][]{\textbf{[#1: #2]}}

%%%%%%%%%%%%%%%%%%%%%%%%%%%%%%%%%%%%%%%%%%%%%%%%%%%%%%%%%%%%%%%%%%%%%%%%%%%%%
%%%%%%%%%%%%%%%%%%%%%%%%%%%%%%%%%%%%%%%%%%%%%%%%%%%%%%%%%%%%%%%%%%%%%%%%%%%%%
\begin{document}

\maketitle


\section{The model}

The input into the P-unit model, $x(t)$, is
\begin{itemize}
\item the fish's own EOD
  \begin{equation}
    \label{eod}
    x(t) = x_{EOD}(t) = \cos(2\pi f_{EOD} t)
  \end{equation}
  with EOD frequency $f_{EOD}$ and amplitude normalized to one.
\item the EOD multiplied with an amplitude modulation $AM(t)$:
  \begin{equation}
    \label{am}
    x(t) = (1+AM(t)) \cos(2\pi f_{EOD} t)
  \end{equation}
  For a random amplitude modulaten ($AM(t) = RAM(t)$) random numbers
  are drawn for each frequency up to $f_{EOD}/2$ in Fourier
  space. After backtransformation the resulting signal is scaled to
  the desired standard deviation relative to the EOD carrier.
\item a superposition of two EODs
  \begin{equation}
    \label{beat}
    x(t) = x_{EOD}(t) + x_{EOD1}(t) = \cos(2\pi f_{EOD} t) + \alpha_1 \cos(2\pi f_{EOD1} t)
  \end{equation}
  with the EOD of the second fish having frequency $f_{EOD1}$ and amplitude $\alpha_1$ relative to the amplitude of the receiving fish.
\item a superposition of many EODs
  \begin{equation}
    \label{multibeat}
    x(t) = x_{EOD}(t) + \sum_{i=1}^{n} x_{EODi}(t) = \cos(2\pi f_{EOD} t) + \sum_{i=1}^{n} \alpha_i \cos(2\pi f_{EODi} t) \; .
  \end{equation}
  For $n=2$ this is our cocktail-party problem.  
\end{itemize}
The input $x(t)$ is a normalized EOD and thus is unitless.

First, the input $x(t)$, is thresholded potentially at the synapse between the receptor cells and the afferent, and then low-pass filtered  with time constant $\tau_{d}$ by the afferent's dendrite:
\begin{equation}
  \label{dendrite}
  \tau_{d} \frac{d V_{d}}{d t} = -V_{d}+  \lfloor x(t) \rfloor_{0}^{p}
\end{equation}
Because the input is unitless, the dendritic voltage is unitless, too. $\lfloor x(t) \rfloor_{0}$ denotes the threshold operation that sets negative values to zero:
\begin{equation}
  \label{threshold}
  \lfloor x(t) \rfloor_0 = \left\{ \begin{array}{rcl} x(t) & ; & x(t) \ge 0 \\ 0 & ; & x(t) < 0 \end{array} \right.
\end{equation}
Usually the exponent $p$ is set to one (pure threshold). In our advanced models $p$ is set to three in order to reproduce responses to beats with difference frequencies above half of the EOD frequency.

This thresholding and low-pass filtering extracts the amplitude modulation of the input $x(t)$. The dendritic voltage $V_d(t)$ is the input to a leaky integrate-and-fire (LIF) model
\begin{equation}
  \label{LIF}
  \tau_{m} \frac{d V_{m}}{d t}  = - V_{m} + \mu + \alpha V_{d} - A + \sqrt{2D}\xi(t)
\end{equation}
where $\tau_{m}$ the membrane time constant, $\mu$ a fixed bias current, $\alpha$ a scaling factor for $V_{d}$, and $\sqrt{2D}\xi(t)$ a white noise of strength $D$. All terms in the LIF are unitless.

The adaptation current $A$ follows
\begin{equation}
  \label{adaptation}
  \tau_{A} \frac{d A}{d t} = - A
\end{equation}
with adaptation time constant $\tau_A$.

Whenever the membrane voltage $V_m(t)$ crosses the threshold $\theta=1$ a spike is generated, $V_{m}(t)$ is reset to $0$, the adaptation current is incremented by $\Delta A$, and integration of $V_m(t)$ is paused for the duration of a refractory period $t_{ref}$:
\begin{equation}
  \label{spikethresh}
  V_m(t) \ge \theta \; : \left\{ \begin{array}{rcl} V_m & \mapsto & 0 \\ A  & \mapsto & A + \Delta A/\tau_A \end{array} \right.
\end{equation}


\section{Parameter values}

\note[JB]{Sascha, list all parameters (table or itemize) plus time step of the model with typical values and the right (time) units}

\begin{table}[h!]
  \begin{center}
    \caption{Model parameters.}
    \label{tab:table1}
    \begin{tabular}{l|c|l}
      
      parameter & explanation & median parameter \\
      \hline
      $\alpha$  & stimulus scaling factor &  90.533695\,cm \\
      $\tau_{m}$  & membrane time constant &  0.001847\,s\\       
      $\mu$  & bias current &  -17.1875\,mV\\      
      $\sqrt{2D}$  & noise strength &  0.01848\,mV$\sqrt{\textup{s}}$ \\      
      $\tau_{A}$  & adaption time constant &  0.111759\,s\\      
      $\Delta A$  & adaption strength &  0.122197\,mVs\\      
      $\tau_{d}$  & time constant of dendritic low-pass filter &  0.002463\,s\\      
      $t_{ref}$  & absolute refractory period &  0.000965\,s\\     
	  $\Delta t$  & time step &  0.00005\,s\\           
    \end{tabular}
  \end{center}
\end{table}

\note[SR]{t Median Parameter sind da. Die Werte sind dann doch sehr klein sollte ich das in Sekunden umrechnen? Auch noch ein Bild der Parameter Vert}
\note[SR]{Die Tabelle ist sehr ähnlich der in der Masterarbeit von Alex (Table 3), die Einheiten in der Tabelle der Masterarbeit scheinen jedoch nicht zu stimmen (ich denke es müssten Sekunden und keine Millisekunden sein). Die Einheiten habe ich mir bei einem Abgleich der Werte der csv Tabelle und dem Parameter Wertebereich in den Bildern der Masterabeit abgeleitet (figure 19).}




\section{Numerical implementation}

\note[JB]{Sascha: This is Alexander Ott's code from the git repository, right?}
\note[SR]{Yes that is correct.}
The ODEs are integrated by the Euler forward method with time-step $\Delta t$.

For the intrinsic noise of the model $\xi(t)$ in each time step $i$ a random number is drawn from a normal distribution $\mathcal{N}(0,\,1)$ with zero mean and standard deviation of one. This number is multiplied with $\sqrt{2D}$ and divided by $\sqrt{\Delta t}$:
\begin{equation}
  \label{LIFintegration}
  V_{m_{i+1}}  = V_{m_i} + \left(-V_{m_i} + \mu + \alpha V_{d_i} - A_i + \sqrt{\frac{2D}{\Delta t}}\mathcal{N}(0,\,1)_i\right) \frac{\Delta t}{\tau_m}
\end{equation}
\note[JB]{Benjamin: wir haben das Rauschen innerhalb der Klammer. Damit ist zwar das $\sqrt{\Delta t}$ richtig, aber wir teilen noch durch $\tau_m$. Damit sind effektiv unsere $D$ Werte andere als wenn wir den Rauschterm ausserhalb der Klammer haetten (so machst du das glaube ich immer).}

The noise strength values from the table in fact equal $\sqrt{2D}$ and not $D$.

\note[JB]{we need to clean up that noise issue}
\note[SR]{Wie genau?}

\end{document}
